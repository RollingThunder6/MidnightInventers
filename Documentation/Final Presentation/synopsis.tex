\documentclass[12pt,a4paper,final]{article}
\usepackage[utf8]{inputenc}
\usepackage{amsmath}
\usepackage{amsfonts}
\usepackage{amssymb}
\usepackage{graphicx}
\usepackage{fancyhdr}
\usepackage{mathptmx}
\usepackage{float}
\usepackage{sectsty}
\usepackage{enumitem}
\usepackage{array}
\sectionfont{\fontsize{12}{15}\selectfont}

\usepackage{etoolbox}
\makeatletter
\patchcmd{\l@section}
  {\hfil}
  {\leaders\hbox{\normalfont$\m@th\mkern \@dotsep mu\hbox{.}\mkern \@dotsep mu$}\hfill}
  {}{}
\makeatother
\usepackage[left=3.5cm,right=1.25cm,top=2.5cm,bottom=1.25cm]{geometry}
\graphicspath{ {Images/} }

\begin{document}
\newgeometry{left=1in,right=1in,top=1in,bottom=1in}
\section*{}
\pagenumbering{gobble}
\begin{center}
\Huge
\textbf{
Detection of DDoS in OpenSDN environment using SVM and Entropy based mechanism.
}

\vspace*{1.5cm}

\large
\textbf{
PROJECT SYNOPSIS
}

\vspace*{1.5cm}
\textbf{
BACHELOR OF ENGINEERING
}

\Large
\textbf{
Computer Engineering
}

\vspace*{1cm}
\large
SUBMITTED BY
\vspace*{1cm}
\linebreak
Achyuth Rao
\linebreak
Akib Shaikh
\linebreak
Arun Pottekat
\linebreak
Pranav Tale
\linebreak

July 2016

\begin{figure}[h]
\begin{center}
\includegraphics[scale=0.25]{logo.png}
\end{center}
\end{figure}

\Large
\textbf{
P. E. S. MODERN COLLEGE OF ENGINEERING,
\linebreak
PUNE
}
\end{center}
\newpage

\tableofcontents
\listoffigures
\listoftables

\newgeometry{left=3.5cm,right=1.25cm,top=2.5cm,bottom=1.25cm}
\section{Title}
\setcounter{page}{1}
\pagenumbering{arabic}
\begin{flushleft}
\normalsize
Detection of DDoS in OpenSDN environment using SVM and Entropy based mechanism.
\linebreak

\noindent
\section{Domain}
Networking, Machine Learning and Physics
\linebreak

\noindent
\section{Keywords}
SDN, SVM, Entropy, DDoS
\linebreak

\noindent
\section{Team}
\begin{quotation}
Group Id: 2 \hfill
\linebreak

Team Members: 
\begin{enumerate}
\item
Achyuth Rao - 41056

\item
Akib Shaikh - 41062

\item
Arun Pottekat - 41054

\item
Pranav Tale - 41070
\end{enumerate}
\end{quotation}

\noindent
\section{Objective}
\begin{enumerate}
\item
To apprehend the different type of network attacks which can be launched on SDN.

\item
To compare different types of DDoS detection mechanisms.

\item
To grasp an overview about the different network monitoring tools.
\end{enumerate}

\noindent
\section{Scope}
\begin{enumerate}
\item
To setup Software Defined Network environment and network monitoring tool to analyse the environment.
\item
To understand the working of Support Vector Machine classifier and Entropy based mechanism in a Software Defined Network environment to detect DDoS attacks.
\item
To analyse the OpenFlow statistics and develop an application to confirm the alert generated.
\end{enumerate}

\noindent
\section{Feasibility Study}
\begin{enumerate}
\item
One can setup SDN environment virtually which is meant for research and testing needs.

\item
Entire project will be based on Open Source technologies and thus numerous resources and documentation is easily available

\item
For training of the SVM many datasets are available, for our project we will be using the dataset provided by DARPA.
\linebreak
DARPA 2000 Scenario Specific dataset.
\end{enumerate}

\noindent
\section{Technical Details}
\begin{quotation}
\textbf{Platform} 
\begin{enumerate}
\item
Software Defined Network

\item
Linux Operating System

\item
OpenFlow Switches
\end{enumerate}

\textbf{Software Specification}
\begin{enumerate}
\item
OpenDayLight Controller - 0.4.2 Berrylium SR2

\item
Oracle VirtualBox

\item
Mininet 2.2.1

\item
POX Controller

\item
Tenzor Flow - 1.4

\item
libsvm

\item
Python 2.7+, 3.0+

\item
Nagios 4

\item
GruntJS

\item
Text Editor
\end{enumerate}

\textbf{Hardware Specification}
\begin{enumerate}
\item
Raspberry Pi Zero

\item
USB to LAN Connectors

\item
Ethernet Cables

\item
System Configuration :- 
\begin{enumerate}
\item
Processor : i3 5th generation Machines

\item
RAM : 4GB
\end{enumerate}
\end{enumerate}

\textbf{Dataset}
\begin{enumerate}
\item
DARPA 2000 Scenario Specific dataset
\end{enumerate}
\end{quotation}

\noindent
\section{Innovativeness and Usefulness}
\begin{enumerate}
\item
Entropy and SVM based applications are light weight and can be run on network devices without clogging it.

\item
Our solution can be deployed in the enterprise networks for enhancing security.

\item
Integrating Machine Learning and Artificial Intelligence with the new norms of networking i.e. SDN.

\item
Run-time alert about DDoS attacks using Nagios.
\end{enumerate}

\noindent
\section{Market Potential and Competitive Advantage}
\begin{enumerate}
\item
Since 2013 there have been many SDN deployements in production as it enables centralized network management.

\item
SDN is estimated to reach approximately \$35 billion by 2018 - Market Landscape Report.

\item
A single DDoS attack can cost a company more than 4 lakh dollars and hence it becomes a necessity to detect such attacks quickly and efficiently.

\item
Integrating applications like SVM and Entropy help to improve the security features of SDN.
\end{enumerate}

\noindent
\section{Brief Description}
\begin{figure}[H]
\begin{center}
\includegraphics[scale=0.5]{Architecture_Diagram.png}
\caption{System Architecture Diagram}
\end{center}
\end{figure}

\begin{figure}[H]
\begin{center}
\includegraphics[scale=0.5]{sdn-architecture.png}
\caption{SDN Architecture Diagram}
\end{center}
\end{figure}
\begin{enumerate}
\item
Software Defined Networks involves seperation of the control plane and data plane.

\item
Forwarding of packets is done in the data plane and
intelligence of the entire network resides in the control plane, making it vulnerable to network attacks.

\item
DDoS attacks which cause heavy utilization of bandwidth must be detected dynamically with high rate of accuracy.

\item
Thus, we compare two solutions for fast and effective detection :- 
\begin{enumerate}
\item
Entropy based Discretization :- Entropy is a measure of the probability of an event happening with reference to the total number of events occurring.

\item
Support Vector Machine Classififer :- is an accurate classifier capable of decision making from uncertain information.
\end{enumerate}

\item
As soon
as the attack is detected a ticket will be raised to the network team by a network monitoring tool like Nagios

\item
It is necessary to verify the alert by viewing the statistics of OpenFlow switches which will be provided as an application running on top of the controller for the
network administrator.
\end{enumerate}

\newpage
\noindent
\section{Major Milestones and Dates}
\begin{table}[H]
\begin{tabular}{ | m{1cm} | m{9cm}| m{3cm} | } 
\hline
\textbf{Sr. No} & \textbf{Module} & \textbf{End Date} \\
\hline
1. & 
Setup of SDN Environment. & 
August 2016 \\ 
\hline

2. &
Monitoring SDN traffic using network monitoring tool to generate ticket in case of detection of DDoS attack. &
March 2017 \\ 
\hline

3. & 
Implementation of Support Vector Machine to detect DDoS attack. &
December 2016 \\
\hline

4. & 
Implementation of OpenFlow statistics monitoring application. &
January 2017 \\
\hline

5. & 
Implementation of Entropy based discretization to detect DDoS attack. &
March 2017 \\
\hline
\end{tabular}
\caption{Major Milestones and Dates}
\end{table}

\noindent
\section{References}
\begin{enumerate}[label={[\arabic*]}]
\item 
"DDoS Detection and Analysis in SDN-based Environment Using Support Vector Machine Classifier" -Kokila RT, S. Thamarai Selvi, Kannan Govindarajan - 2014 Sixth International Conference on Advanced Computing(ICoAC) - Department of Computer Technology, Anna University (MIT Campus), Chennai

\item 
"An Entropy-Based Distributed DDoS Detection Mechanism in Software-Defined Networking" - Rui Wang, Zhiping Jia, Lei Ju - 2015 IEEE Trustcom/BigDataSE/ISPA - School of Computer Science and Technology Shandong University Jinan, China

\item 
"Software-Defined Networking:The New Norm for Networks and 
Open Networking Foundation" - Open Networking Foundation - ONF White Paper April 13, 2012

\item 
"Detection of DDoS Attacks using Enhanced Support Vector Machines with Real Time Generated Dataset" - T.Subbulakshmi , Dr. S. Mercy Shalinie, V.GanapathiSubramanian, K.BalaKrishnan, D. AnandK, K.Kannathal - IEEE-ICoAC 2011 - Department of CSE, TCE Madurai, India.

\item
"OpenFlow Switch Specification" - Open Networking Foundation - Version 1.3.2 2013
\end{enumerate}
\end{flushleft}
\end{document}