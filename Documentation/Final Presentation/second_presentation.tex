\documentclass[10pt]{beamer}
\usetheme{Warsaw}
\usepackage[utf8]{inputenc}
\usepackage[english]{babel}
\usepackage{amsmath}
\usepackage{amsfonts}
\usepackage{amssymb}
\usepackage{wrapfig}
\usepackage{graphicx}
\usepackage{array}
\graphicspath{ {Images/} }
\author{
\footnotesize
\textbf{
\linebreak Achyuth Rao - 41056
\linebreak Akib Shaikh - 41062
\linebreak Arun Pottekat - 41054
\linebreak Pranav Tale - 41070
\linebreak 
\linebreak
Guide:- Prof. Mrs. Aparna Junnarkar
}
}

\title{
\small
\textbf{
Detection of DDoS in SDN environment using SVM and Entropy based mechanism.
}
}
\institute{
\includegraphics[width=2.75cm, height=2.9cm]{logo.png}\\ \textbf{PES's Modern College Of Engineering}} 
\date{}
\setbeamertemplate{footline}[frame number]{}
\setbeamertemplate{navigation symbols}{} 
\begin{document}

\begin{frame}
\titlepage
\end{frame}


\begin{frame}
\frametitle{Problem Statement}
\begin{itemize}
\footnotesize
\item
To provide a solution for the detection of DDoS attack in SDN environment using SVM and Entropy based mechanism and monitoring OpenFlow statistics.
\end{itemize}
\end{frame}







\begin{frame}
\frametitle{Motivation}
\begin{center}
\begin{itemize}
\footnotesize
\item 
The centralized controller is a potential single point of attack.
\item
The Southbound interface, OpenFlow is vulnerable to threats.
\item
DDoS attack renders an online service unavailable by overloading it.
\item
Thusly, there is a need to optimally detect DDoS in SDN.
\end{itemize}
\end{center}

\end{frame}



%\begin{frame}
% \frametitle{Literature Survey}
% \scriptsize
% \begin{center}
% \begin{tabular}{ | m{2cm} | m{2cm}| m{2cm} | m{3cm} | } 
% \hline
% \textbf{Title} & \textbf{Author} & \textbf{Journal and Year} & \textbf{Description} \\
% \hline
% DDoS Detection and Analysis in SDN-based Environment Using Support Vector Machine Classifier & 
% Kokila RT, S. Thamarai Selvi, Kannan Govindarajan & 
% IEEE 2014 & 
% This paper provides information about DDoS attack in SDN environment using Support Vector Machine to classify the attack.\\ 
% \hline
% An Entropy-Based Distributed DDoS Detection Mechanism in Software-Defined Networking &
% Rui Wang, Zhiping Jia, Lei Ju & 
% IEEE 2015 & 
% This paper provides information about DDoS attack in SDN environment using Entropy based mechanism to classify the attack.\\ 
% \hline
% Software-Defined Networking:The New Norm for Networks & 
% Open Networking Foundation & 
% ONF White Paper, 2012 & 
% Description about Software Defined Networks\\
% \hline
% Detection of DDoS Attacks using Enhanced Support Vector Machines with Real Time Generated Dataset & 
% T.Subbulakshmi, Dr. S. Mercy Shalinie, D. AnandK,  K.Kannatha&
% IEEE 2013 &
% Provided information how to create and use datasets for SVM.\\
% \hline
% OpenFlow Switch Specification& 
% Open Networking Foundation &
% Version 1.3.2 2013 &
% Description about OpenFlow Protocol\\
% \hline
% \end{tabular}
% \end{center}
% \end{frame}

\begin{frame}
\frametitle{Objective}
\begin{center}
\begin{itemize}
\footnotesize
\item
To apprehend different types of network attacks which can be launched on SDN.
\item
To compare different types of DDoS.
\item
To grasp an overview about the different network monitoring tools.
\end{itemize}
\end{center}
\end{frame}

\begin{frame}
\frametitle{Scope}
\begin{center}
\begin{itemize}
\footnotesize
\item
%OpenFlow protocol in SDN switches.
Set up of SDN environment.
\item
Entropy and SVM based DDoS detection method.
\item
OpenFlow Monitoring application using OpenDaylight API.
\end{itemize}
\end{center}
\end{frame}



%\begin{frame}
%\frametitle{Synopsis}
%\begin{center}
%\begin{itemize}
%\footnotesize
%\item
%SDN involves seperation of control and data plane.

%\item
%Security is a major concern in SDN architecture.

%\item
%DDoS attack results in exhaustion of controller resources.

%\item
%Entropy is a good measure of randomness.

%\item
%SVM is capable of decision making from uncertain information.

%\item
%Application for viewing OpenFlow statistics.
%\end{itemize}
%\end{center}
%\end{frame}


% \begin{frame}{
% \small
% \textbf{
% Scope
% }
% }
% \begin{itemize}
% \footnotesize
% \item
% Implementation of:
% \begin{itemize}
% \footnotesize
% \item
% Entropy mechanism.

% \item
% Support Vector Machine classifier.

% \item
% OpenFlow statistics monitoring application.
% \end{itemize}
% \end{itemize}
% \end{frame}

% \begin{frame}{
% \small
% \textbf{
% Architecture Diagram
% }
% }
% \begin{figure}[H]
% \includegraphics[scale=0.25]{Architecture_Diagram.png}
% \caption{
% System Architecture
% }
% \end{figure}
% \end{frame}


\begin{frame}
\frametitle{Literature Survey}
\scriptsize
\begin{center}
\begin{tabular}{ | m{2cm} | m{2cm}| m{2cm} | m{3cm} | } 
\hline
\textbf{Title} & \textbf{Author} & \textbf{Journal and Year} & \textbf{Description} \\
\hline
DDoS Detection and Analysis in SDN-based Environment Using Support Vector Machine Classifier & 
Kokila RT, S. Thamarai Selvi, Kannan Govindarajan & 
IEEE 2014 & 
This paper provides information about DDoS attack in SDN environment using Support Vector Machine to classify the attack.\\ 
\hline
An Entropy-Based Distributed DDoS Detection Mechanism in Software-Defined Networking &
Rui Wang, Zhiping Jia, Lei Ju & 
IEEE 2015 & 
This paper provides information about DDoS attack in SDN environment using Entropy based mechanism to classify the attack.\\ 
\hline
Software-Defined Networking:The New Norm for Networks & 
Open Networking Foundation & 
ONF White Paper, 2012 & 
Description about Software Defined Networks\\
\hline
Detection of DDoS Attacks using Enhanced Support Vector Machines with Real Time Generated Dataset & 
T.Subbulakshmi, Dr. S. Mercy Shalinie, D. AnandK,  K.Kannatha&
IEEE 2013 &
Provided information how to create and use datasets for SVM.\\
\hline
OpenFlow Switch Specification& 
Open Networking Foundation &
Version 1.3.2 2013 &
Description about OpenFlow Protocol\\
\hline
\end{tabular}
\end{center}
\end{frame}



\begin{frame}
\frametitle{Architecture Diagram}
\begin{figure}[H]
\includegraphics[scale=0.38]{myarch.png}
\caption{System Architecture}
\end{figure}
\end{frame}

\begin{frame}
\frametitle{Mathematical Model}
\end{frame}

\begin{frame}
\frametitle{Algorithmic Strategies}
\end{frame}

	

\begin{frame}
\frametitle{Software Specifications}
\begin{itemize}
\footnotesize
\item
Linux based Operating System.
\item
OpenDayLight Controller - 0.4.2 Berrylium SR2.
\item
Oracle VirtualBox.
\item
Mininet 2.2.1
\item
POX Controller.
\item
Tenzor Flow 1.4
\item
LibSVM.
\item
Python 2.7 or above.
\item
Nagios Core.
\item
ReactJS
\end{itemize}
\end{frame}

\begin{frame}
\frametitle{Hardware Specifications}
\begin{itemize}
\footnotesize
\item
Raspberry Pi Zero Controller.
\item
USB to LAN Connectors
\item
Ethernet Cables
\item
Zodiac FX OpenFlow Switch.
\end{itemize}
\end{frame}



\begin{frame}
\frametitle{Dataset Specifications}
\begin{itemize}
\footnotesize
\item
"DDoS attack 2007" dataset provided by the Center for Applied Internet Data Analysis(CAIDA).
\item
The 1998 DARPA's network traffic dataset provided by MIT Lincoln Lab.
\item
The 2000 DARPA intrusion detection scenario specific dataset provided by MIT Lincoln Lab which contains:
\end{itemize}

\begin{table}
\scriptsize
\caption{2000 DARPA Dataset details}
\begin{center}
\begin{tabular}{ | m{2cm} | m{2cm}| m{2cm} |} 
\hline
\textbf{Data Category} & \textbf{No. of training instances} & \textbf{No. of test instances} \\
\hline
Break In &
156 &
374 \\
\hline
DDoS &
963 &
1035 \\
\hline
Installsw &
318 &
204 \\
\hline
IPSweep &
101 &
684 \\
\hline
Normal &
2500 &
2501 \\
\hline
Probe &
54 &
94 \\
\hline
Total &
4092 &
4892 \\
\hline
\end{tabular}
\end{center}
\end{table}
\end{frame}



\begin{frame}
\frametitle{Results of Entropy Based Discretization}
\begin{itemize}
\footnotesize
\item
Machine with Ubuntu 14.04, i5 CPU and 8G RAM.
\item
Mininet as a network simulator (Tree Topology, 800Mbps Link speed, 20 hosts).
\item
Open vSwitch.
\item 
Floodlight controller.
\item
CAIDA's "DDoS Attack 2007" dataset.
\end{itemize}
\scriptsize
\vspace{0.1cm}
\begin{table}
\caption{parameter values of the Traffic}
\begin{center}
\begin{tabular}{ | m{2cm} | m{2cm}| m{2cm} |} 
\hline
\textbf{S. No} & \textbf{Average Traffic Rate(Mbps)} & \textbf{Attack Rate(pkts/s)} \\
\hline
Exp.1 &
50 &
50-200 \\
\hline
Exp.2 &
100 &
300-500 \\ 
\hline
Exp.3 &
500 &
1000-2000 \\
\hline
\end{tabular}
\end{center}
\end{table}
\begin{figure}[h]
\includegraphics[scale=0.2]{Entropy.png}
\caption{The normalized entropy value of IPdst Flow}
\end{figure}

\end{frame}

\begin{frame}
\frametitle{Results of SVM based Method}

\begin{itemize}
\footnotesize
\item
%The 2000 DARPA intrusion detection scenario specific dataset %provided by MIT Lincoln lab is taken for evaluation.
The normal traffic data is included from 1998 DARPA dataset.
\item

The attack traffic data is included from 2000 DARPA dataset.
\end{itemize}

\begin{table}
\scriptsize
\caption{Accuracy with different parameters}
\begin{center}
\begin{tabular}{ | m{2cm} | m{2cm}| m{2cm} | m{2cm} |} 
\hline
\textbf{Cost} & \textbf{Gamma} & \textbf{Classification Accuracy(\%)} & \textbf{False Positive} \\
\hline
10 &
0.1 &
94.23 &
0.011 \\
\hline
10 &
0.01 &
95.11 &
0.008 \\
\hline
10 &
0.001 &
93.86 &
0.013 \\
\hline
\end{tabular}
\end{center}
\end{table}

\begin{figure}[h]
\includegraphics[scale=0.35]{svm.png}
\caption{Camparison of classification methods	}

\end{figure}
\end{frame}



\begin{frame}
\frametitle{Conclusion}

\begin{itemize}
\footnotesize
\item
Taking into consideration the advantages of SDN, security issues need to be resolved.
\item
This project will be a step towards enhancing the security in SDN which will soon replace the traditional networks.
\end{itemize}
\end{frame}




% \begin{frame}
% \frametitle{Conclusion}
% \begin{center}
% \begin{itemize}
% \item
% \end{itemize}
% \end{center}
% \end{frame}
\begin{frame}
\frametitle{References}
\begin{itemize}
\footnotesize
\item
"DDoS Detection and Analysis in SDN-based Environment Using Support Vector Machine Classifier" -Kokila RT, S. Thamarai Selvi, Kannan Govindarajan - 2014 Sixth International Conference on Advanced Computing(ICoAC) - Department of Computer Technology, Anna University (MIT Campus), Chennai.

\item
"An Entropy-Based Distributed DDoS Detection Mechanism in Software-Defined Networking" - Rui Wang, Zhiping Jia, Lei Ju - 2015 IEEE Trustcom/BigDataSE/ISPA - School of Computer Science and Technology Shandong University Jinan, China.

\item
"Software-Defined Networking:The New Norm for Networks and 
Open Networking Foundation" - Open Networking Foundation - ONF White Paper April 13, 2012.

\item
T.Subbulakshmi , Dr. S. Mercy Shalinie, V.GanapathiSubramanian, K.BalaKrishnan, D. AnandK, K.Kannathal - IEEE-ICoAC 2011 - Department of CSE, TCE Madurai, India.

\item
"OpenFlow Switch Specification" - Open Networking Foundation - Version 1.3.2 2013.

\end{itemize}
\end{frame}



\begin{frame}{}
\Huge
Thank You\ldots
\end{frame}
\end{document}